
    \documentclass[a4paper]{report}
    \pagestyle{plain}
    %\usepackage{amsmath}
    %\usepackage{amssymb}
    %\usepackage{amsthm}
    %\usepackage{fancyvrb}
    \usepackage[usenames]{color}
    \usepackage[linkcolor=black,citecolor=black]{hyperref}
    \usepackage{makeidx}
    \makeindex

    \setlength{\parindent}{0cm}
    \addtolength{\parskip}{0.5em}
    %\renewcommand{\thefootnote}{\fnsymbol{footnote}}

    \begin{document}
    \input{defun.tex}

    \title{ATDOC example project}
    %\date{}
    \maketitle

    \tableofcontents
    \newpage

    
    \chapter{The example/app package}
    This is docstring for the \textbf{example/app} package.


                   The package contains a function which does it's job by
                   applying transformation to the first and second arguments:

                   

    \rule{\linewidth}{0.1mm}
    
    \label{example/app__fun__foo}
    \begin{defun}[Function]
    foo first &key (other 100500)


    
    \bigskip
    \textsc{Arguments}

first
	--- Just a first argument.

other
	--- Optional keyword argument. Default is 100500.




    
    \bigskip
    \textsc{Return Values}

A string with first and other concatenated.


	
    \bigskip
    \textsc{Details}

This is example function.         

   Internally it calls \hyperref[example/utils__fun__do-the-job]{\texttt{example/utils:do-the-job}}
  
   to do the real job.


   Note, that the link above is broken, but Coo does not warn us when building the docs.
   Sphinx issues a warning inn such case.


      
    \bigskip
    \textsc{See also}


	
    \begin{itemize}
    
	  
    \item
    \hyperref[example/utils__fun__do-the-job]{example/utils:do-the-job}
    
	
    \end{itemize}
  
      


    
    \end{defun}
  
  


                   When you mention a function like that, it is included into
                   the package description and removed from the                   "Other functions..." section
    \chapter{The example/class package}
    This package demonstrates how ATDOC displays classes and generic functions.


                   The key consept is user class:

                   

    \rule{\linewidth}{0.1mm}
    
    \label{example/class__class__user}
    \begin{defun}[Class]
    user


      
    \bigskip
    \textsc{Superclasses}

\color[rgb]{0.5,0.5,0.5}common-lisp:standard-object\color[rgb]{0,0,0}, \color[rgb]{0.5,0.5,0.5}sb-pcl::slot-object\color[rgb]{0,0,0}, \color[rgb]{0.5,0.5,0.5}common-lisp:t\color[rgb]{0,0,0}


      
    \bigskip
    \textsc{Documented Subclasses}

\hyperref[example/class__class__admin]{
	  admin
	}
      , \hyperref[example/class__class__non-documented-user]{
	  non-documented-user
	}
      


	
    \bigskip
    \textsc{Direct Slots}

email --- Correct email address.

last-login-at --- 

name --- A full username.




	
    \bigskip
    \textsc{Details}

All users in the system have this class.


Last login slot is updated automatically.

\textbf{NOTE:} "Documented Subclasses" section contains only classes which are:


    
    \end{defun}
  
  


                   It is possible to check if user has admin privileges, using this function:

                   

    \rule{\linewidth}{0.1mm}
    
    \label{example/class__fun__is-admin}
    \begin{defun}[Function]
    is-admin user


	
    \bigskip
    \textsc{Details}

Returns t if user can modify the system.


    
    \end{defun}
  
  

                   Right now, \hyperref[example/class__fun__is-admin]{\texttt{is-admin}}
   returns \texttt{t} only for objects of \hyperref[example/class__class__admin]{\texttt{admin}}
  :

                   

    \rule{\linewidth}{0.1mm}
    
    \label{example/class__class__admin}
    \begin{defun}[Class]
    admin


      
    \bigskip
    \textsc{Superclasses}

\hyperref[example/class__class__user]{
	  user
	}
      , \color[rgb]{0.5,0.5,0.5}common-lisp:standard-object\color[rgb]{0,0,0}, \color[rgb]{0.5,0.5,0.5}sb-pcl::slot-object\color[rgb]{0,0,0}, \color[rgb]{0.5,0.5,0.5}common-lisp:t\color[rgb]{0,0,0}


      
    \bigskip
    \textsc{Documented Subclasses}


	    None
	  


	
    \bigskip
    \textsc{Direct Slots}


	      None
	    


	
    \bigskip
    \textsc{Details}

Admins should have additional priveleges.


    
    \end{defun}
  
  
      \section{Other classes}
      

    \rule{\linewidth}{0.1mm}
    
    \label{example/class__class__non-documented-user}
    \begin{defun}[Class]
    non-documented-user


      
    \bigskip
    \textsc{Superclasses}

\hyperref[example/class__class__user]{
	  user
	}
      , \color[rgb]{0.5,0.5,0.5}common-lisp:standard-object\color[rgb]{0,0,0}, \color[rgb]{0.5,0.5,0.5}sb-pcl::slot-object\color[rgb]{0,0,0}, \color[rgb]{0.5,0.5,0.5}common-lisp:t\color[rgb]{0,0,0}


      
    \bigskip
    \textsc{Documented Subclasses}


	    None
	  


	
    \bigskip
    \textsc{Direct Slots}


	      None
	    


	No documentation string.  Possibly unimplemented or incomplete.
	


    
    \end{defun}
  
  
    \chapter{The example/utils package}
    The utils package.

This package's docstring is not mention any functions via \texttt{aboutfun} or \texttt{aboutclass} tags.

Hence, the only exported function \hyperref[example/utils__fun__do-the-job]{\texttt{do-the-job}}
   will be show in a separate
section "Other functions in example/utils".
      \section{Other functions}
      

    \rule{\linewidth}{0.1mm}
    
    \label{example/utils__fun__do-the-job}
    \begin{defun}[Function]
    do-the-job first second


	
    \bigskip
    \textsc{Details}

The function does the job.

It concatenates first and second arguments
calling internal function concat.


On this multiline we'll check how does documentationsystem processes docstrings.


    
    \end{defun}
  
  

    \printindex
    \end{document}
  