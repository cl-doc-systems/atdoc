
    \documentclass[a4paper]{report}
    \pagestyle{plain}
    %\usepackage{amsmath}
    %\usepackage{amssymb}
    %\usepackage{amsthm}
    %\usepackage{fancyvrb}
    \usepackage[usenames]{color}
    \usepackage[linkcolor=black,citecolor=black]{hyperref}
    \usepackage{makeidx}
    \makeindex

    \setlength{\parindent}{0cm}
    \addtolength{\parskip}{0.5em}
    %\renewcommand{\thefootnote}{\fnsymbol{footnote}}

    \begin{document}
    % defun library.
% Guy L. Steele Jr.
% Copyright 1985, 1986 , 1988, 1989 Guy L. Steele Jr.

%  3/18/85 12:48  Now \@defunname generates index entries.
%  5/22/85 14:16  Use \noexpand before \tt and \rm in index entries.
%		  This will delay their processing until the point
%		  when the index is read back in again.
%  5/23/85 17:01  Add special check for nil as an init value for an
%		  &optional or &key argument; make it be \tt, not \it.
% 12/17/85 12:56  Add "lisp" environment.
%  1/02/86 14:15  Base "lisp" environment on tabbing instead of raggedright.
%  1/02/86 17:00  Make \endlisp use \ignorespaces.
%  1/03/86 17:00  Put a space into the index key to make sorting work better.
%  1/31/86 12:32  Simplify "lisp" environment to let the tabbing environment
%		  take care of interparagraph spacing.
%  1/31/86 13:27  Add \smallcaps stuff.
%  1/31/86 13:38  In \@defunkeymode et al., assume the user provides a :.
%  1/31/86 16:03  Flush \smallcaps stuff.
%  1/31/86 17:03  Use \addvspace in \defun et al.
%  2/04/86 17:15  Add feature so that the syntax \begin{defun}[Frob] causes "[Frob]"
%		  to appear at the right margin, as in the Common Lisp book, and causes
%		  the index entry for function baz to be "baz frob".  If no square
%		  brackets ae used, then nothing appears at the right margin, and the
%		  index entry looks like "baz \lowercase{default}" where "default" is
%		  the definition of \defaultdefuntype, which is initialized by this file
%		  to {Function} but can be redefined by the user.
%  2/07/86 17:08  Make lisp environment use \frenchspacing.
%  3/11/86 11:42  Improve some of the breaking places in the defun headers:
%		  Introduce \hbox around every name, and make better pseudo-hyphens.
%		  Also fix a bug in \@defuninitvalue.
%  3/12/86 11:47  Hair up the defunbreak stuff some more.  Now we have different
%		  behavior depending on whether the function name is long or short.
%  3/12/86 16:09  Fix a possible bug in \@showdefuntype.
%  3/13/86 14:16  More fixes to hairy defunbreak stuff.
%  5/29/86 01:47  Add \internalroutine.
%  6/04/86 14:47  Correct spacing in \@showdefuntype.
%  6/29/88 00:31  Remove priority-box macros.
%
% This file patches some problems with using defun.sty without other
% Guy magic.
%

\def\cf{\tt\frenchspacing}

\catcode`@=11	% Make it possible to refer to some LaTeX utility macros.
\catcode`&=11

\def\lisp{\cf\tabbing}
\let\endlisp=\endtabbing

\def\@defunstart{\noindent\leavevmode     % Need this to trigger the \everypar for margin rules.
  \begingroup
  \samepage  \raggedright \cf \catcode`&=11
  \hyphenpenalty=-5  \exhyphenpenalty=\@M  \brokenpenalty=\@M}

% The addvspace used to be "plus 4pt" but that was not enough stretch relative to \parskip.
\def\defun{\@ifstar{\@defunstart{}\@ifnextchar[{\@defuntyped}{\@defununtyped}}{\relax
   \addvspace{18pt plus 12pt minus 6pt}\@defunstart    %Margin macros know this amount
   \@ifnextchar[{\@defuntyped}{\@defununtyped}}}


\def\@showdefuntype{\setbox0\hbox{\hskip1.5em$\lbrack$\it\@defuntype\/$\rbrack$}\relax
		    \setbox1\hbox{\hskip\textwidth\hskip-1\wd0}\relax
		    \leavevmode\parshape=2  0pt 1\wd1  0pt \textwidth  \relax
		    \hbox to 0pt{\hbox to \textwidth{\hss\box0}\hss}\let\@defuntype\@defunsecondtype}

\def\@defununtyped{\let\@defunshow=\relax \@defun}
\def\@defuntyped[#1]{\def\@defuntype{#1}\let\@defunsecondtype\@defuntype
                     \let\@defunshow=\@showdefuntype
                     \@ifnextchar[{\@defuntypedtwice}{\@defun}}
\def\@defuntypedtwice[#1]{\def\@defunsecondtype{#1}\let\@defunshow=\@showdefuntype
                          \@defun}

\def\@defun{\@ifnextchar?{\@defunname}{\@defunname}}	% Skips spaces before name 

\def\enddefun{\par
  %%\message{end{defun} for \currentdefun}%For debugging
  \@ifstar{}{\addvspace{6pt plus 6pt minus 2pt}}}    %Margin macros know this amount
% Used to be "plus 2pt" but that was not enough stretch relative to \parskip.

\def\@defunname#1 {\@defunshow
  #1\@defunindex#1:\par{\lowercase{\@defuntype}}\relax
  %%\message{begin{defun} for #1}\global\def\defun@name{#1}%For debugging
  \setbox0\hbox{#1}\relax
  \ifdim 1\wd0 < 1.3in\relax
     \def\@defunbreak{\discretionary{\hbox{}}{\setbox0\hbox{#1 }\hbox{\hskip 1\wd0}}{\hbox{ }}}\relax
     \let\@firstdefunbreak=~\relax
  \else
     \def\@defunbreak{\discretionary{\hbox{}}{\hbox{\hskip2em}}{\hbox{ }}}\relax
     \let\@firstdefunbreak=\@defunbreak
  \fi
  \def\@defunkeywordbreak{\@firstdefunbreak\let\@defunkeywordbreak=\@defunbreak}\relax
  \def\@hairydefunbreak{\@firstdefunbreak
		        \let\@defunkeywordbreak=\@defunbreak
		        \let\@hairydefunbreak\@defunbreak}\relax
  \@defunreqmode}

\def\@defunindex#1:#2\par#3{\def\@tempa{}\def\thing{#2}\ifx\thing\@tempa
  \@idefunindex{}#1:\par{#3}\else \@idefunindex{#1:}#2\par{#3}\fi}
\def\@idefunindex#1#2:\par#3{\index{{#2 }{\noexpand\noexpand\noexpand\cf #1#2\ \noexpand\noexpand\noexpand\rm #3}}}


\def\@newlinecheck#1#2{\def\@tempa{#1}\def\@tempb{\\}\ifx\@tempa\@tempb \\*\let\@tempd\@defunname
  \else \def\@tempd{#2}\fi \@tempd}

\def\@ifparnext#1#2{\def\@tempa{#1}\def\@tempb{#2}\futurelet\@tempc\@ifparnx}
\def\@ifparnx{\ifx\@tempc\@@par \let\@tempd\@tempa \else \let\@tempd\@tempb \fi \@tempd}


\def\@defunreqmode{\@ifparnext{\@defundone}{\@ifnextchar &{\@defunkeyword}{\@ifnextchar({\@defunparenreqarg}{\@defunreqarg}}}}
\def\@defunreqarg#1 {\@newlinecheck{#1}{\@hairydefunbreak\hbox{\it #1\/}\@defunreqmode}}
\def\@defunparenreqarg(#1 #2) {\@hairydefunbreak(\hbox{\it #1\/}~#2)\@defunreqmode}  %CLOS methods

\def\@defunrestmode{\@ifparnext{\@defundone}{\@ifnextchar &{\@defunkeyword}{\@defunrestarg}}}
\def\@defunrestarg#1 {\@newlinecheck{#1}{\penalty\@m\ \hbox{\it #1\/}\@defunrestmode}}

\def\@defunoptionalmode{\@ifparnext{\@defundone}{\@ifnextchar &{\@defunkeyword}{\@ifnextchar({\@defunparenoptionalarg}{\@defunoptionalarg}}}}
\def\@defunoptionalarg#1 {\@newlinecheck{#1}{\@hairydefunbreak\hbox{\it #1\/}\@defunoptionalmode}}
\def\@defunparenoptionalarg(#1 #2) {\@hairydefunbreak(\hbox{\it #1\/}~\@defuninitvalue#2)\@defunoptionalmode}

\def\@defuninitvalue#1#2){{\ifcat#1A\def\@tempa{#1#2}\def\@tempb{nil}\ifx\@tempa\@tempb
  \cf \else \it \fi \else \cf \fi #1#2})}

\def\@defunkeymode{\@ifparnext{\@defundone}{\@ifnextchar &{\@defunkeyword}{\@ifnextchar({\@defunparenkeyarg}{\@defunkeyarg}}}}
\def\@defunkeyarg#1 {\@newlinecheck{#1}{\penalty\@m\@defunkeywordbreak\hbox{#1}\@defunkeymode}}
\def\@defunparenkeyarg(#1 #2) {\penalty\@m~(\hbox{#1}~\@defuninitvalue#2)\@defunkeymode}

\def\@defunkeyword &#1 {\@defunkeywordbreak\hbox{\&#1}\csname @defun#1mode\endcsname}

\def\@defundone\par{\par\endgroup
   %%\edef\currentdefun{\defun@name}%For debugging
   \nobreak\addvspace{6pt}\noindent}



\begingroup
\catcode`\[=13 \catcode`\]=13 \catcode`\(=13 \catcode`\)=13
\catcode`\<=13 \catcode`\>=13 \catcode`\|=13 \catcode`\?=13
\global\def\@defmacstart#1{\relax
%  \noindent\leavevmode     % Need this to trigger the \everypar for margin rules.
  \begingroup \samepage  \topsep 0pt
  \catcode`&=11
  \def\@lbracehack{{$\,\lbrace$}\pushtabs\=\+\@backslashsetup\it}\relax
  \def\@rbracehack{\-\poptabs
       \@ifnextchar*{\@rbracesuper}{\@ifnextchar+{\@rbracesuper}{\/$\rbrace\,$\@backslashsetup\it}}}\relax
  \def[{{$\,\lbrack$}\pushtabs\=\+\@backslashsetup\it}\relax
  \def]{\-\poptabs$\/\rbrack\,$\@backslashsetup\it}\relax
  \def<{{$\,\dlbrack$}\pushtabs\=\+\@backslashsetup\it}\relax
  \def>{\-\poptabs$\/\drbrack\,$\@backslashsetup\it}\relax
  \def({{\cf \char40}\pushtabs\=\+\@backslashsetup\it}\relax
  \def){\-\poptabs{\cf \char41}\@backslashsetup\it}\relax
  \def|{{$\char124$}}\relax
  \def?{{$\downarrow$}}\relax
  \def\@finishdefmac{\-\poptabs\end{tabbing}\endgroup\everypar{}
       \addvspace{6pt}\noindent}\relax
  \expandafter\def\@carret{\expandafter\@ifnextchar\@carret
     {\@finishdefmac\@gobblecr}{\@tabcr*\@backslashsetup\it}}\relax
  \@defunhackbraces
  \def\@backslashsetup{\def\\{\-\poptabs\@tabcr*\@backslashsetup
				\@dodefmacname{#1}}}\relax
  \catcode`\[=13 \catcode`\]=13 \catcode`\(=13 \catcode`\)=13
  \catcode`\<=13 \catcode`\>=13 \catcode`\|=13 \catcode`\?=13 \catcode`\^^M=13
  \def\!##1!{\cd{##1}}\relax
  \begin{tabbing}\@backslashsetup\@margineveryparguts
  \@dodefmacname{#1}}
\endgroup


\def\Mopt#1{{$\,\lbrack$}{\it #1\/}{$\rbrack\,$}}
\def\Mchoice#1{{$\,\dlbrack$}{\it #1\/}{$\drbrack\,$}}
\def\Mstar#1{{$\,\lbrace$}{\it #1\/}{$\rbrace^*\,$}}
\def\Mplus#1{{$\,\lbrace$}{\it #1\/}{$\rbrace^+\,$}}
\def\Mgroup#1{{$\,\lbrace$}{\it #1\/}{$\rbrace\,$}}
\def\Mor{{$|$}}
\def\Mind#1{$\downarrow${\it #1\/}}

% The following is taken from /usr/local/lib/tex/macros/latex.tex
% but corrects a scoping bug.

\newdimen\@curtabmardimen
\def\@startline{\global\@curtabmar\@nxttabmar\relax
   \global\@curtabmardimen\dimen\@curtabmar
   \global\@curtab\@curtabmar\global\setbox\@curline\hbox
    {}\@startfield\strut}

\def\@stopline{\unskip\@stopfield\if@rjfield \global\@rjfieldfalse
   \@tempdima\@totalleftmargin \advance\@tempdima\linewidth
   \hbox to\@tempdima{\@itemfudge\hskip\@curtabmardimen
   \box\@curline\hfil\box\@curfield}\else\@addfield
   \hbox{\@itemfudge\hskip\@curtabmardimen\box\@curline}\fi}

% End of material from /usr/local/lib/tex/macros/latex.tex .

\def\defmac{\@defmacbegin{Macro}}
\let\enddefmac\enddefun

\def\defspec{\@defmacbegin{Special form}}
\let\enddefspec\enddefun

\def\defloop{\@defmacbegin{Loop clause}}
\let\enddefloop\enddefun

\def\@defmacbegin#1{\@ifstar{\@defmacstart{#1}}{\addvspace{18pt plus 12pt minus 6pt}\@defmacstart{#1}}}

{\catcode`\^^M=13\global\def\@carret{
}\global\def\@defmacnamecrgobble#1
{\@defmacname{#1}}}

\def\@dodefmacname#1{\cf\expandafter\@ifnextchar\@carret
   {\@defmacnamecrgobble{#1}}{\@defmacname{#1}}}

\def\@defmacname#1#2 {%\typeout{#1: #2 }\relax
  \setbox0\hbox{\hskip1.5em$\lbrack$\it#1\/$\rbrack$}\relax
  \leavevmode\hbox to 0pt{\hbox to \textwidth{\hss\box0}\hss}{\cf #2\ }\pushtabs\=\+\@backslashsetup
  \@defunindex#2:\par{\lowercase{#1}}
  \it}

\bgroup
\catcode`\<=1 \catcode`\>=2 \catcode`\{=13 \catcode`\}=13
\global\def\@defunhackbraces<\catcode`\{=13\catcode`\}=13\let{\@lbracehack\let}\@rbracehack>
\egroup

\def\@rbracesuper#1{{\/$\rbrace^{#1}\,$}\@backslashsetup\it}


\catcode`@=12		% Restore character codes
\catcode`&=4

%[End]



    \title{ATDOC example project}
    %\date{}
    \maketitle

    \tableofcontents
    \newpage

    
    \chapter{The example/app package}
    This is docstring for the \textbf{example/app} package.


                   The package contains a function which does it's job by
                   applying transformation to the first and second arguments:

                   

    \rule{\linewidth}{0.1mm}
    
    \label{example/app__fun__foo}
    \begin{defun}[Function]
    foo first &key (other 100500)


    
    \bigskip
    \textsc{Arguments}

first
	--- Just a first argument.

other
	--- Optional keyword argument. Default is 100500.




    
    \bigskip
    \textsc{Return Values}

A string with first and other concatenated.


	
    \bigskip
    \textsc{Details}

This is example function.         

   Internally it calls \hyperref[example/utils__fun__do-the-job]{\texttt{example/utils:do-the-job}}
  
   to do the real job.


   Note, that the link above is broken, but Coo does not warn us when building the docs.
   Sphinx issues a warning inn such case.


      
    \bigskip
    \textsc{See also}


	
    \begin{itemize}
    
	  
    \item
    \hyperref[example/utils__fun__do-the-job]{example/utils:do-the-job}
    
	
    \end{itemize}
  
      


    
    \end{defun}
  
  


                   When you mention a function like that, it is included into
                   the package description and removed from the                   "Other functions..." section
    \chapter{The example/class package}
    This package demonstrates how ATDOC displays classes and generic functions.


                   The key consept is user class:

                   

    \rule{\linewidth}{0.1mm}
    
    \label{example/class__class__user}
    \begin{defun}[Class]
    user


      
    \bigskip
    \textsc{Superclasses}

\color[rgb]{0.5,0.5,0.5}common-lisp:standard-object\color[rgb]{0,0,0}, \color[rgb]{0.5,0.5,0.5}sb-pcl::slot-object\color[rgb]{0,0,0}, \color[rgb]{0.5,0.5,0.5}common-lisp:t\color[rgb]{0,0,0}


      
    \bigskip
    \textsc{Documented Subclasses}

\hyperref[example/class__class__admin]{
	  admin
	}
      , \hyperref[example/class__class__non-documented-user]{
	  non-documented-user
	}
      


	
    \bigskip
    \textsc{Direct Slots}

email --- Correct email address.

last-login-at --- 

name --- A full username.




	
    \bigskip
    \textsc{Details}

All users in the system have this class.


Last login slot is updated automatically.

\textbf{NOTE:} "Documented Subclasses" section contains only classes which are:


    
    \end{defun}
  
  


                   It is possible to check if user has admin privileges, using this function:

                   

    \rule{\linewidth}{0.1mm}
    
    \label{example/class__fun__is-admin}
    \begin{defun}[Function]
    is-admin user


	
    \bigskip
    \textsc{Details}

Returns t if user can modify the system.


    
    \end{defun}
  
  

                   Right now, \hyperref[example/class__fun__is-admin]{\texttt{is-admin}}
   returns \texttt{t} only for objects of \hyperref[example/class__class__admin]{\texttt{admin}}
  :

                   

    \rule{\linewidth}{0.1mm}
    
    \label{example/class__class__admin}
    \begin{defun}[Class]
    admin


      
    \bigskip
    \textsc{Superclasses}

\hyperref[example/class__class__user]{
	  user
	}
      , \color[rgb]{0.5,0.5,0.5}common-lisp:standard-object\color[rgb]{0,0,0}, \color[rgb]{0.5,0.5,0.5}sb-pcl::slot-object\color[rgb]{0,0,0}, \color[rgb]{0.5,0.5,0.5}common-lisp:t\color[rgb]{0,0,0}


      
    \bigskip
    \textsc{Documented Subclasses}


	    None
	  


	
    \bigskip
    \textsc{Direct Slots}


	      None
	    


	
    \bigskip
    \textsc{Details}

Admins should have additional priveleges.


    
    \end{defun}
  
  
      \section{Other classes}
      

    \rule{\linewidth}{0.1mm}
    
    \label{example/class__class__non-documented-user}
    \begin{defun}[Class]
    non-documented-user


      
    \bigskip
    \textsc{Superclasses}

\hyperref[example/class__class__user]{
	  user
	}
      , \color[rgb]{0.5,0.5,0.5}common-lisp:standard-object\color[rgb]{0,0,0}, \color[rgb]{0.5,0.5,0.5}sb-pcl::slot-object\color[rgb]{0,0,0}, \color[rgb]{0.5,0.5,0.5}common-lisp:t\color[rgb]{0,0,0}


      
    \bigskip
    \textsc{Documented Subclasses}


	    None
	  


	
    \bigskip
    \textsc{Direct Slots}


	      None
	    


	No documentation string.  Possibly unimplemented or incomplete.
	


    
    \end{defun}
  
  
    \chapter{The example/utils package}
    The utils package.

This package's docstring is not mention any functions via \texttt{aboutfun} or \texttt{aboutclass} tags.

Hence, the only exported function \hyperref[example/utils__fun__do-the-job]{\texttt{do-the-job}}
   will be show in a separate
section "Other functions in example/utils".
      \section{Other functions}
      

    \rule{\linewidth}{0.1mm}
    
    \label{example/utils__fun__do-the-job}
    \begin{defun}[Function]
    do-the-job first second


	
    \bigskip
    \textsc{Details}

The function does the job.

It concatenates first and second arguments
calling internal function concat.


On this multiline we'll check how does documentationsystem processes docstrings.


    
    \end{defun}
  
  

    \printindex
    \end{document}
  